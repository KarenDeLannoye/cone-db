% !TEX root = ConeDB_Technote.tex

\section{Introduction} \label{sec:intro}
\subsection{Background} \label{ssec:intro-background}
Need for fire testing:
\begin{itemize}
\item Bench-scale apparatus to measure HRR
\item scientific/material property basis for flammability response (not just response-to-fire testing)
\item Quantify burning behavior under controlled conditions
\item Material flammability, relative ranking
\item xyz?
\end{itemize}

``The instrument described in 1982 did not yet have the instrumentation for measuring smoke production, which was later developed with the help of George Mulholland and was published in 198720. When the Cone Calorimeter was thus completed, it received the prestigious R\&D 100 award in 1988 for being one of the 100 most important engineering inventions of the year.''

\subsection{Conception of the NIST Cone Calorimeter Database} \label{ssec:intro-concept}
Since its development at NIST in 1982 (then the National Bureau of Standards, NBS)~\cite{babrauskas1982nbs}, the cone calorimeter has enabled researchers to collect invaluable data for the characterization of material flammability behavior in response to well-characterized heating conditions. This data – including measurements of sample and product mass loss rate (MLR), heat release rate (HRR; based on oxygen consumption calorimetry), and smoke and toxic gas production – has been crucial in the development of fire safety standards and fire-resistant materials. Currently, this wealth of measurement data is presented across dozens of separate publications, making it difficult to effectively access, parse, and use.

This project aims to create a digital repository of cone calorimetry data, organized with standard formatting and metadata to allow easy access to experimental measurements and derived properties (e.g., heat of combustion, ignition and fire growth parameters, and average burning rate in response to known heating conditions). These values will be compiled and stored in a publicly accessible, version controlled, and continually growing database for use by fire safety scientists and engineers outside NIST. Data from 1983 to the present will be made available, including thousands of experiments on a wide range of materials including commodity plastics, woods, cables, vegetative fuels, and flooring and other building materials. Additionally, the database is designed to integrate with the Fire Dynamics Simulator, FDS, by presenting measured material property information with standard formatting needed for use as model inputs in FDS. 

\subsection{Connection to } \label{ssec:intro-databases}

Intro the Matl-Fl-DB. \\
Connect to FCD for formatting/design.\\
Connect to FDS Validation guide.
